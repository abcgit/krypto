\section*{Thesen Kontra}

\subsection*{Key Escrow}
Das Key Escrow Verfahren basiert auf einer zentrale Instanz,
die die jeweiligen Schlüssel sammelt und verwaltet.
Durch diese zentrale Stelle gehen mehrere Gefahren aus. Zum einen ist eine
solche Stelle ein sehr attraktives Ziel von Kriminellen, auf der
anderen Seite kann ein Einbruch in eine für verschiedene weitere
kriminelle Tätigkeiten weiterverwendet werden, wie z.B.
Wirtschaftsspionage oder verschiedene Arten von Betrug.
Ein anderes Gefahrenpotential geht von den Angestellten dieser Stelle aus oder
sogar vom Betreiber (wie z.B. der Staat).
Es ist auch bei scheinbar vertrauenswürdigen Instanzen möglich,
dass die Daten veräussert werden und an Unbefugte gelangen.

\subsection*{Vorratsdatenspeicherung}
Ein häufiges Argument der Befürworter einer Vorratsdatenspeicherung ist, dass ja nur unter gewissen Umständen der Zugriff auf solche Daten erlaubt sei. Ein richterlicher Beschluss oder ähnliches ist immer nötig um solche Daten in einem Verfahren verwenden zu dürfen.

Das mag zwar alles stimmen. Das Problem ist aber, wenn die Daten einmal gespeichert sind, hat der Bürger keine Kontrolle mehr darüber. Was hindert ein Staat daran, die Gesetze anzupssen, so dass diese plötzlich länger gespeichert werden oder auch bei leichtem Verdacht bereits verwendet werden dürfen? Die Daten sind ja schon da ...

\subsection*{Sicherheit der Daten}
Eine Überwachung wie wir sie heute kennen generiert eine grosse Menge an Daten. Damit diese verwendet werden können, müssen diese gespeichert werden. Dies führt jedoch zu einigen Problem.

Die gespeicherten Daten sind nicht nur für den Staat interessant. Diese können auch für kriminelle Zwecke verwendet werden. Und da die Daten nun an einem zentralen Ort gespeichert werden, macht diesen zu einem lukrativen Ziel. Somit stellt sich die Frage, können die gespeicherten Daten überhaupt genug sicher abgelegt werden?

Es zeigt sich immer wieder, dass selbst grosse Unternehmen Ziel von Angriffen werden. Es gibt erschreckende Beispiele:
\begin{itemize}
\item RSA Security, ein Hersteller von Sicherheitssoftware
\item Lockeed Martin, ein Rüstungsunternehmen und Lieferant der amerikanischen Armee
\item Mitsubishi Heavy Industries, ebenfalls ein Rüstungsunternehmen und Lieferant der amerikanischen Armee
\end{itemize}

Die Dunkelziffer wird sehr gross sein. Aber nur schon die bekannt gewordenen Fälle sind Grund zur Beunruhigung. Das zeigt doch, wie proffessionell Kriminelle heutzutage vorgehen.

Es ist durchaus auch vorstellbar, dass solche Angriffe nicht nur krimineller Energie entstammen, sondern auch durch verfeindete Staaten durchgeführt werden. Cyber-War wird wohl eine immer grössere Bedeutung haben.

Es stellt sich also die Frage, wer alles auf diese Daten Zugriff hat! Ein ungutes Gefühl ...


\subsection*{Videoüberwachung}
Die Videoüberwachung mag ein veritables Mittel gegen die Verbrechensbekämpfung sein. Jedoch wirkt die Abschreckung nur in überwachten Gebieten. Das heisst, die Kriminalität verlagert sich in andere Gebiete und kann dort dafür umso brutaler werden. Diesem Umstand könnte man nur mit einer totalen Überwachung entgegenwirken, was jedoch tunlichst zu vermeiden ist!

Ein weiterer wichtiger Punkt der in Betracht gezogen werden muss: Die Freiheit des Individuums wird eingeschränkt. Es wird einfacher, Personen auszugrenzen. Es wird einfacher, Randgruppen der Gesellschaft noch mehr an den Rand zu drängen. Anonymität erlaubt es Personen, welche mit einem sozialen Stigma behaftet sind, z.B. Aidskranke, Alkoholiker o.a. sich auszutauschen, ohne von der Gesellschaft ausgegrenzt und benachteiligt zu werden.

\subsection*{Bundestrojaner}
Es kann nicht garantiert werden, dass eine installierter Bundestrojaner nur
von den befugten Behörden verwendet wird. Untersuchungen haben gezeigt, dass
auch das Stück Software fehlerhaft sein kann und so von unbefugten Dritten für
den Zugriff auf ein System des Verdächtigen erlaubt. Dies erlaubt z.B. das
Unterschieben von falschen Informationen welche dann fälschlicherweise gegen
den Verdächtigen benutzt werden können.



% Ideas:

%These:
%(Totale) (Video-) Überwachung verhindert Verbrechen, da sie es erlaubt, die Täter zu verfolgen und zur Rechenschaft zu ziehen. Dies hat eine Abschreckende Wirkung.
%
%Gegenthese:
%Die Kriminalität verlagert sich in andere Bereiche. Wichtiger: Die Freiheit des Individuums wird eingeschränkt. Es wird einfacher, Personen auszugrenzen. Es wird einfacher, Randgruppen der Gesellschaft noch mehr an den Rand zu drängen.
%
%These:
%Anonymität verhindert, dass Kriminelle verfolgt werden.
%
%Antithese:
%Anonymität erlaubt es Personen, welche mit einem sozialen Stigma behaftet sind, z.B. Aidskranke, Alkoholiker o.a. sich auszutauschen, ohne von der Gesellschaft ausgegrenzt und benachteiligt zu werden.
%
%These:
%Ohne Sicherheit gibt es keine Freiheit, da dann die Gesellschaft/Demokratie zerfällt.
%
%Antithese:
%Zuviel Sicherheit und Überwachung führen zu einem totalitären Staat.
%
%These:
%Gesammelte Daten werden zum Wohl der gesammten Gesellschaft verwendet.
%
%Antithese:
%Die gesammelten Daten werden zum Nachteil des einzelnen Individuums verwendet, z.B. bei Erbkrankheiten -> Versicherungsprämie. Ausserdem sind die wirklichen Profiteure z.B. die Pharmakonzerne, die die verwendeten Daten zum Entwickeln neuer Medikamente verwenden, welche dann teuer verkauft werden. 

%Terror Bekämpfung
%Pro:
%Ohne Überwachung ist ein Staat anfälliger auf terroristische Angriffe und
%Attentate. Eine Überwachung der Kommunikationsmittel hilft solche Angriffe
%frühzeitig zu erkennen und nötige Gegenmassnahmen einzuleiten. Falls ein
%solches Verbrechen trotzdem erfolgreich sein sollte, können mögliche Täter
%gefunden werden.
%
%Kontra:
%Eine Überwachung die vor terroristischen Angriffen schützen soll muss alle
%Kommunikationskanäle abdecken. Sobald es einen nichtüberwachten Kanal gibt,
%wird sich die Kommunikation über diesen Kanal stattfinden.
%Eine Überwachung aller Kommunikationskanäle ist jedoch nicht nur beinahe
%unbezahlbar, auch ist es technisch beinahe unmöglich jeden Kommunikationskanal zu
%überwachen.
%Eine solche Überwachung führt zu einem totalitären Staat und wäre nur möglich
%wenn jegliche Verschlüsselung verboten wäre.
%
%
%Key Escrow
%Pro:
%Durch dein Einsatz eines Key Escrow Verfahrens kann eine
%Strafverfolgungs-Behörde bei zwingendem Verdacht jegliche Verschlüsselten
%Daten entschlüsseln. Dies kann die Strafverfolgung in vielen Fällen
%effizienter gestalten wie z.B. in Fällen wie: Besitz von Kinderpornographie,
%
%Kontra:
%Ein Key Escrow Verfahren bedingt eine zentrale Stelle, welche die jeweiligen
%Schlüssel sammelt.
%Durch diese zentrale Stelle gehen mehrere Gefahren aus. Zum einen ist eine
%solche Stelle ein sehr attraktives Ziel von Kriminellen. Ein Einbruch in eine
%solche Stelle kann für verschiedene weitere kriminelle Tätigkeiten
%weiterverwendet werden wie z.B. Wirtschaftsspionage oder verschiedene Arten
%von Betrug.
%Ein anderes Gefahrenpotential geht von den Angestellten dieser Stelle aus oder
%sogar von Staat direkt. Es ist ein Trugschluss zu glauben, dass die Daten
%korrekt behandelt werden und nicht an Unbefugte gelangen.
%
%
%
%Bundestrojaner
%Kontra:
Es kann nicht garantiert werden, dass eine installierter Bundestrojaner nur
von den befugten Behörden verwendet wird. Untersuchungen haben gezeigt, dass
auch das Stück Software fehlerhaft sein kann und so von unbefugten Dritten für
den Zugriff auf ein System des Verdächtigen erlaubt. Dies erlaubt z.B. das
Unterschieben von falschen Informationen welche dann fälschlicherweise gegen
den Verdächtigen benutzt werden können.
