\section{
Glossar
}

\subsection{
Big Brother 
}
Imagin\"are Person aus dem von George Orwell 1949 ver\"offentlichten 
Buch "1984". Big Brother symbolisiert den \"Uberwachungsstaat und dessen 
totalit\"are Auswirkungen. 

\subsection{
Key Escrow 
}
Vom Staat verordnete Speicherung s\"amtlicher zur Verschl\"usselung 
ben\"otigter kryptographischer Schl\"ussel. Dies hat zur Folge, dass 
Kryptographie ohne Hinterlegung der Schl\"ussel illegal sein muss. 

\subsection{
DNA 
}
Die DNA (Deoxyribonucleic acid) ist ein Molek\"ul, das die 
Erbinformationen beinhaltet. Sie erm\"oglicht es, durch Spuren, z.B. 
Haare, Personen einigermassen sicher zu identifizieren. Im Erbgut des 
Menschen sind vererbbare Krankheiten resp. Veranlagungen f\"ur dieselben 
wie z.B. Alzheimer ersichtlich. 

\subsection{
Privatsph\"are 
}
Die Privatsph\"are einer nat\"urlichen Person ist ein im Gesetz 
festgeschriebenes Prinzip, welches als Menschenrecht betrachtet wird und 
juristisch unter gewissen Bedingungen eingeschr\"ankt werden kann, z.B. 
im Strafrecht oder bei Personen \"offentlichen Interesses. Es soll der 
Person Freiheiten einberaumen, damit sie sich pers\"onlich Entfalten 
kann, ohne Repressionen oder Einschr\"ankungen f\"urchten zu m\"ussen. 
Privatsph\"are spielt heutzutage sowohl in der realen Welt ("physisch") 
als auch - und dies in zunehmendem Masse - im Internet eine Rolle 
("virtuell"). 

\subsection{
Bundestrojaner 
}
Bundestrojaner im weiteren Sinn bezeichnet eine Art Software, welche auf 
dem Computer einer zu \"uberwachenden Person installiert wird. Es geht 
dabei darum, verschl\"usselte Daten und Datenstr\"ome vor der 
Verschl\"usselung oder nach der Entschl\"usselung zu lesen. Eingesetzt 
wird der Bundestrojaner von Strafverfolgungsbeh\"orden. 

\subsection{
Verschl\"usselung / Kryptographie (?) 
}
Umwandlung von Daten in eine Form, die von Drittpersonen nicht ohne 
Kenntnis einer weiteren Information, des sogenannten Schl\"ussels, 
gelesen werden kann. Im Falle von Kommunikation m\"ussen die beiden 
Parteien auf irgendeine Art und Weise den Schl\"ussel austauschen, damit 
sie miteinander verschl\"usselt kommunizieren k\"onnen. Man 
unterscheidet "starke" und "schwache" Verschl\"usselung; stark meint in 
diesem Zusammenhang, dass bei sorgf\"altigem Umgang die Daten nicht ohne 
den Schl\"ussel entschl\"usselt werden k\"onnen. 

\subsection{
Telekommunikations\"uberwachung (TK\"U) 
}
Telekommunikations\"uberwachung (Englisch: Lawful Interception) 
bezeichnet die vom Staat durch Legislation befohlene M\"oglichkeit, 
Ermittlungs- und anderen Beh\"orden Zugriff auf Verkehrsdaten und/oder 
Dateninhalte von Telekommunikationsinfrastruktur zu erm\"oglichen. 
Verkehrsdaten (auch Verbindungsdaten) beinhalten z.B. Zeitpunkt, Ort, 
Teilnehmer und Dauer eines Telefongespr\"achs, nicht aber den 
Dateninhalt, also die gesprochenen Worte. Die Analyse von Verkehrsdaten 
wird auch Verkehrsflussanalyse (Englisch: Traffic Analysis) genannt. 

\subsection{
Chip 
}
Eigentlich Mikrochip, wird im allgemeinen Sprachgebrauch f\"ur 
hochintegrierte, kleinste elektronische Schaltungen verwendet. Chips 
k\"onnen zur Datenerfassung, -verarbeitung, -speicherung und zur 
Kommunikation verwendet werden. 

\subsection{
Echtzeit\"uberwachung 
}
Damit ist der unmittelbare Zugriff auf durch \"Uberwachung gewonnene 
Daten gemeint. Z.B. k\"onnen Ermittler bei einem Telefongespr\"ach 
gleichzeitig mit den Teilnehmern mith\"oren. 

\subsection{
Vorratsdatenspeicherung 
}
Das Erheben und Speichern von Daten ohne Verdacht, f\"ur die sp\"atere 
Verwendung von Strafverfolgungs- und anderen Beh\"orden. Sie wird oft im 
Gesetz mit Mindest- oder Maximalspeicherdauer festgelegt. 

\subsection{
Cyberkrieg 
}
Ein Begriff, dessen Bedeutung noch nicht eindeutig festgelegt worden 
ist. Wird oft als Erweiterung des konventionellen Krieges auf 
Informationsnetzwerke und dessen virtuellen R\"aume gesehen. Die 
m\"oglichen Vorgehensweisen reichen von Propaganda bis zur Sabotage, 
welche durchaus in der realen Dimension Auswirkungen haben kann. 

\subsection{
Video\"uberwachung 
}
Das kontinuierliche Filmen von Orten durch Videokameras und oft 
Speicherung der aufgezeichneten Daten. Die Video\"uberwachung wird zur 
Verbrechenspr\"avention und nachtr\"aglichen Ermittlung von T\"atern 
eingesetzt. So die Theorie. 

\subsection{
Verkehrsflussanalyse / Traffic Analysis (s. auch TK\"U) 
}
Im engeren Sinne bezeichnet der Begriff die Analyse von Verkehrsdaten 
mit dem Ziel, Bewegungsmuster und Netzwerke von Personen oder 
Personengruppen zu identifizieren. Es geht darum, aus einzelnen Zeit-, 
Ort- und Personen-Datenpunkten ein abstraktes Bild zu erzeugen. 

\subsection{
Datenschutz 
}
Unter Datenschutz versteht man den Schutz personenbezogener Daten, d.h. 
Daten, die u.a. die Privatsp\"are einer Person betreffen. Der 
Datenschutz ist in Europa auf Gesetzesebene geregelt und soll 
sicherstellen, dass jede einzelne Person \"uber ihre eigenen Daten 
bestimmen kann. Dabei ist es irrelevant, ob die datensammelnde 
Institution der Staat oder eine Firma ist. 
