\documentclass{scrartcl}
\usepackage[utf8]{inputenc}
\usepackage[ngerman]{babel}
\usepackage{graphicx}
\usepackage{url}
\usepackage{amsmath}
\usepackage{float}
%\usepackage{hyperref}
\usepackage{array}
\usepackage{amssymb} % doppeltes Z, N und R mit \mathbb{Z} für ganzzahlig.
\usepackage{verbatim} %um Kommentare einzufügen
%\usepackage[a4paper,vmargin={30mm,25mm},hmargin={20mm,20mm}]{geometry}
%\usepackage{multirow}

\begin{document}
\section*{99 Gramm Überwachung. Darf es noch etwas mehr sein?}
% Schweizerdeutsch?
"`Wieviel hätten Sie denn gerne?"', wird man sie schon
häufiger an der Käsetheke gefragt haben. Genüsslich haben
Sie noch einen Blauschimmelkäse dazu genommen. Mehr ist ja
schliesslich besser. Oder nicht?

Wieviel Überwachung sollte ein Staat vornehmen? Wann ist
der \textbf{Big Brother} eine Hilfe, wann eine Gefahr?
Handelt es sich überhaupt um \textit{einen} Überwachungsstaat,
oder ist Orwells Vision der Privatisierung zum Opfer gefallen?

Am 42.42.2042 können Sie Zeuge einer Podiumsdiskussion auf
höchstem Niwo werden - bekannte Vertreter aus Politik, Wirtschaft
und den Universitäten finden sich zusammen, um mit Ihnen
zu diskutieren.

Der Abend wird von dem prominenten Moderator Anton Rudi Müller
geleitet. Herr Müller, renommierter Sprachwissenschafter, Politologe
und Mathematikprofessor an der Universität Basel, hat uns in
einem Interview bereits einen kleinen Einblick auf die 
bevorstehende Podiumsdiskussion nächste Woche gegeben:

\begin{quote}
\textbf{NZZ:} Herr Müller, die angekündigte Podiumsdiskussion zum Thema
"`Der Staat als Big Brother?"' hat im Vorfeld bereits zu
heftigen Reaktionen geführt, unter anderem von Herrn
Thomas Würgler, dem Kommandanten der KaPo Zürich, der
sich, so wörtlich "`[...] nicht vorstellen kann, dass jemand
an der Notwendigkeit und der Nützlichkeit einer umfassenden
Überwachung zum Wohle der Bürger zweifeln könne. [...]"'
Wie kam es zu dieser Reaktion und wo sehen sie die Motivation
um dieses Thema in einer Podiumsdiskussion zur Sprache zu
bringen?
\end{quote}
\begin{quote}
\textbf{Müller:} Zunächst einmal möchte ich ankündigen, dass
der von Ihnen zitierte Herr Würgler, nicht zuletzt aufgrund
seiner Erfahrungen aus dem Berufsumfeld, einer unserer Top-Referenten
an diesem Abend sein wird. Der von Ihnen zitierte Abschnitt stammt
bekanntlich aus der \textbf{TALKTÄGLICH}-Sendung vom September,
in der er mit Herrn Hanspeter Thür,
dem eidgenössischen Datenschutzbeauftragter, über die zukünftige
Entwicklung der Videoüberwachung auf öffentlichen Toiletten
diskutierte. 
Der Lehrstuhl für Gesellschaftswissenschaften der Universität Basel
hat diesen Disput, der bei vielen Menschen gemischte Gefühle
hervorgerufen hat, zum Anlass genommen, eine Podiumsdiskussion
zu veranstalten. um das Für und Gegen, Stand und Entwicklung,
sowie Gefahren und Nutzen der Überwachung zu eruieren.
\end{quote}
\begin{quote}
\textbf{NZZ:} Haben Sie vielen Dank für die kurze Übersicht,
Herr Müller.
\end{quote}
\begin{quote}
\textbf{Müller:} Gern geschehen.
\end{quote}

In diesem Sinne überlassen wir Ihnen die Frage, liebe Leser,
"`Wieviel hätten Sie denn gerne?"' und freuen uns Sie am
42.42.2042 in der Universität Basel an der Podiumsdiskussion
begrüssen zu können.


\end{document}

