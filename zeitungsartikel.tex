\documentclass{scrartcl}
\usepackage[utf8]{inputenc}
\usepackage[ngerman]{babel}
\usepackage{graphicx}
\usepackage{url}
\usepackage{amsmath}
\usepackage{float}
%\usepackage{hyperref}
\usepackage{array}
\usepackage{amssymb} % doppeltes Z, N und R mit \mathbb{Z} für ganzzahlig.
\usepackage{verbatim} %um Kommentare einzufügen
%\usepackage[a4paper,vmargin={30mm,25mm},hmargin={20mm,20mm}]{geometry}
%\usepackage{multirow}

\begin{document}
\section*{99 Gramm Überwachung. Darf es noch etwas mehr sein?}
% Schweizerdeutsch?
"`Wieviel hätten Sie denn gerne?"', wird man sie schon
häufiger an der Käsetheke gefragt haben. Genüsslich haben
Sie noch einen Blauschimmelkäse dazu genommen. Mehr ist ja
schliesslich besser. Oder nicht?

Wieviel Überwachung sollte ein Staat vornehmen? Wann ist
der \textbf{Big Brother} eine Hilfe, wann eine Gefahr?
Handelt es sich überhaupt um \textit{einen} Überwachungsstaat,
oder ist Orwells Vision der Privatisierung zum Opfer gefallen?

Am 42.42.2042 können Sie Zeuge einer Podiumsdiskussion aller
Ehren werden - bekannte Vertreter aus Politik, Wirtschaft
und den Universitäten finden sich zusammen, um mit Ihnen
zu diskutieren.

"`Wieviel hätten Sie denn gerne?"' werden 


\end{document}

