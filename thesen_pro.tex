\section*{Thesen PRO}\subsection*{"'Terror Bekämpfung 1"'; Pressesprecher der US-Amerikanischen Botschaft in der Schweiz}Ohne Überwachung ist ein Staat anfällig auf terroristische Angriffe und Attentate. Eine Überwachung der Kommunikationsmittel hilft, Angriffe frühzeitig zu erkennen und nötige Gegenmassnahmen einzuleiten. Es ist wichtig, dass international zusammengearbeitet wird und die Informationen untereinander ausgetauscht werden können. Ich verstehe nicht, weshalb viele Politiker diesen Austausch verhindern wollen. Die Verbrecher und Terroristen schlafen auch nicht: Falls wir die enge Zusammenarbeit nicht weiterentwickeln, werden sie vernetzter sein als wir. Das ist das Gefährlichste was uns passieren kann! Man sieht ja schon heute, wie eng gewisse mafiöse Strukturen zusammenarbeiten und global agieren. Sollten die Verbrecher den Kampf um die Informationen und Daten gewinnen, haben wir verloren. Dieser Entwicklung müssen wir mit allen Mitteln entgegentreten!\\Beispielsweise durch den Einsatz eines Key Escrow Verfahrens kann eine Strafverfolgungsbehörde bei zwingendem Verdacht jegliche verschlüsselten Daten entschlüsseln. Dies kann die Strafverfolgung in vielen Fällen effizienter gestalten. Der Besitz von Kinderpornographie würde ebenso aufgedeckt wie das Vorbereiten von terroristischen Anschlägen.\\Die USA wird oftmals dafür angegriffen (verbal), dass die Einreisebestimmungen streng geregelt sind. Das stimmt, ist aber klar Teil unserer Strategie, alle nötigen Informationen zu sammeln, um einem terroristischen Verbrechen entgegenwirken zu können.\subsection*{"'Terror Bekämpfung 2"'; entweder Politiker von rechts oder Kommandant der Kantonspolizei Kanton XY}Ich bin der Meinung, dass die verfügbaren Mittel, um Verbrechen aufklären zu können (oder noch besser, um sie verhindern zu können) unbedingt ausgeschöpft werden müssen. Die Überwachung von Kommunikationsmittel, wie auch die Videoüberwachung im öffentlichen Raum sind eine grosse Hilfe im Durchsetzen des geltenden Gesetzes. Es geht doch nicht an, dass wir beispielsweise in einer Bahnhofsunterführung Kameras installiert haben, aber die Bilder danach trotzdem nicht verwendet werden dürfen! Obwohl es ganz klar ist, das der Dieb gefilmt wurde während seiner Tat. Oder dass einer zu Hause umgebracht wird und die Bilder des Nachbarns, welcher den Täter zufälligerweise gefilmt hat, als Beweismaterial nicht anerkannt werden. Im schlimmmsten Fall wird dann noch der Nachbar bestraft, weil er ohne Befugnis gefilmt hat! Da bekommt man den Eindruck, einige Leute wollen die Arbeit der Polizei massiv erschweren!\\Ich bin der Überzeugung, dass insbesondere Kameras im öffentlichen Raum ein Gefühl der Sicherheit verbreiten und viele Bösewichte von ihren Taten abhalten. Vor allem die Jugendlichen sollten lernen, dass sie sich anpassen müssen und das wir sie im Auge behalten. Hier haben die Kameras eine gesunde, abschreckende Wirkung.\\Ich frage mich, was die Menschen, welche sich gegen den Ausbau von Überwachung stellen, zu verbergen haben. Der Staat will ja nur Gutes für uns. Wer nichts Illegales tut, hat auch nichts zu befürchten. Bürger, welche gegen die Überwachung sind, sind gegen das Gesetz! Im Endeffekt unterstützen sie Pädophile, Drogenhändler, Mafiosi - das ganze illegale Gesindel! Sie stellen sich gegen unsere Demokratie und gegen das Volk.\subsection*{"'Wir müssen mit der Zukunft gehen"'; Pressesprecher von Google}Google wird oftmals dafür kritisiert, dass wir Informationen sammeln. Ich sehe das ganz anders. Wir sind ein extrem zukunftsträchtiges, innovatives Unternehmen! In Zukunft wird nicht mehr Geld, sondern Information die Welt führen! Die Staaten sollten nicht gegen, sondern mit uns arbeiten. Sie würden von unserem Wissen profitieren können. Ich verstehe nicht, weshalb sie sich gegen eine Entwicklung stellen, welche nicht mehr aufzuhalten ist. Es geht nicht mehr darum, diese Bewegung stoppen, sondern sich ihr zu bedienen und darin der Beste zu sein. Es können nur die gewinnen, welche mit der Zukunft gehen!\\Ich möchte das an einem Beispiel verdeutlichen: Wir sind technisch in der Lage, Millionen von DNA-Datensätzen zu speichern, auszuwerten und miteinander zu vergleichen. Die Medizin könnte daraus wertvolle Informationen gewinnen, welche unglaubliche Fortschritte möglich machen würde. Wir würden Lösungen finden, welche heute noch unvorstellbar sind! Die Medizin könnte innert kürzester Zeit riesige Schritte vorwärtsmachen, wenn diese Daten allen Forschern zur Verfügung gestellt würden. Aber leider wird dies noch immer bekämpft und von Politikern und bestimmten Organisationen verhindert. Wir lassen uns von Bedenken hindern, dabei spielt hier die Zukunft! Je schneller wir uns anpassen, desto schneller profitieren wir davon.\\Wenn wir unser Wissen zusammentragen, können wir unser Leben viel angenehmer gestalten. Jeder nervt sich über Werbung, die er nicht haben will. Mir persönlich gefällt die Vorstellung, nur noch zu mir passende Werbung zu erhalten. Das wäre doch super - ich werde nur noch mit Informationen konfrontiert, die mich interessieren. Das wollen wir doch alle!\\Ich bin der Meinung, dass sich die Auffassung von Privatsphäre stark verändern wird. Informationen, welche heute als privat und persönlich (also schützenswert) betrachtet werden, werden in Zukunft zum Wohle der Gesellschaft geteilt. So können alle vom Wissen der anderen profitieren.\subsection*{"'Sicherheit und Freiheit gehören zusammen"';Professor für Soziologie und neue Medien von der Uni XY}Der Mensch kann sich nur dann frei fühlen, wenn er sich auch sicher fühlt. %In der heutigen Gesellschaft drohen sehr viele Gefahren. Es ist die Aufgabe des Staates, seine Bürger zu (be-)schützen und ihnen das Gefühl von Sicherheit zu geben. Mit der massiven Zunahme von zugänglichen Informationen, sinkt unser Sicherheitsgefühl. Von allen Seiten nehmen wir Bedrohungen war. Wir lesen über Verbrechen wie Diebstähle, sexuelle Missbräuche und Morde in unserer nächsten Umgebung. Wir sehen im Fernsehen die Kriege der ganzen Welt in Echtzeit und so weiter. Wir haben sehr viele Informationen, welche schwierig sind, einzuschätzen. Diese Entwicklung werden wir nicht aufhalten können. Wir müssen uns ihr anpassen und so auch der Staat. Dieser muss Lösungen finden, wie unser Sicherheitsgefühl intakt bleiben kann und gestärkt wird. Eine Lösung ist meiner Meinung nach der Einsatz von Kameras. Wenn diese überall präsent sind, fühlt man sich beschützt. Man ist dann nicht allein und weiss, dass ein Übeltäter beobachtet würde.\\Ein anderes Beispiel ist das Überwachen von Datenverkehr im Internet. Als Vater zweier Töchter wäre mir wohler, wenn ich wüsste, dass in allen Chatrooms ein paar Polizisten mitlesen und im Notfall eingreifen.\\Es ist extrem wichtig, dass Kinder in einer sicheren Umgebung aufwachsen. Wir alle tragen hier eine besondere Verantwortung. Die technischen Entwicklungen schreiten rasant voran, es stehen in immer kürzeren Zeit immer mehr Informationen zur Verfügung. Wir müssen versuchen, vorauszudenken. Es wäre fatal, wenn unsere Kinder alle diese Möglichkeiten nutzen können, aber die Sicherheit dabei verloren ginge.\\Ich denke die Regierung ist gut beraten, wenn sie die Überwachung ausbaut. Ansonsten droht ihr ein Vertrauensverlust von seiten der Bevölkerung.\subsection*{"'Ein Super-Chip für alle"'; durchgeknallter Arzt, der zuviele Medikamente ausprobiert hat}Ich träume davon, dass in Zukunft jeder von uns einen kleinen Chip auf sich trägt. Dieser würde unser Bewegungsprofil, unsere getätigten Einkäufe, medizinische Informationen und vieles mehr speichern. Ein solcher Chip würde uns Ärzten ganz neue Möglichkeiten bringen. Wir könnten den Patienten viel gerechter und individueller behandeln. Wenn einer mit Vergiftungserscheinungen eingeliefert würde, wüssten wir, was er in den letzten Tagen eingekauft (und somit gegessen) hat. Wir könnten viel effizienter reagieren. Wenn jemand verunfallt, wüssten wir mit Hilfe des Chips, welche Blutgruppe die Person hat, welche Medikamente von ihr eingenommen werden, u.s.w. Wenn jemand im Ausland behandelt würde, könnten diese Informationen auch hier wieder ausgelesen und in die Behandlung miteinbezogen werden.\\Bei Unfällen in unbewohntem Gebiet könnte anhand des Bewegungsprofils sofort herausgefunden werden, wo sich die Person genau befindet. Natürlich kann man dies heute beispielsweise bei Schneetouren-Fahrern auch. Aber es wäre sinnvoll, wenn alle Menschen davon profitieren würden und sie sich darüber gar keine Gedanken machen müssten.\\Natürlich brächte ein solcher Chip noch viel mehr Vorteile mit sich: Wenn Eltern ihr Kind vermissen, besteht immer die Möglichkeit es zu orten und es vor einem möglichen Verbrechen zu bewahren. Ich denke hier an eine Entführung. \\Zudem könnte man auch Bankdaten auf dem Chip hinterlegen, damit man bargeldlos und ohne sich einen Pin merken zu müssen bezahlen könnte. Stellen Sie sich vor, man hat diesen Chip im kleinen Finger implantiert und muss bei der Migros-Kasse nur kurz mit dem kleinen Finger über einen Scanner fahren! Das würde uns bestimmt auch das mühsame Anstehen an der Kasse ersparen.