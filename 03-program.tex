\documentclass{scrartcl}
\usepackage[utf8]{inputenc}
\usepackage[ngerman]{babel}
\usepackage{graphicx}
\usepackage{url}
\usepackage{amsmath}
\usepackage{float}
%\usepackage{hyperref}
\usepackage{array}
\usepackage{amssymb} % doppeltes Z, N und R mit \mathbb{Z} für ganzzahlig.
\usepackage{verbatim} %um Kommentare einzufügen
%\usepackage[a4paper,vmargin={30mm,25mm},hmargin={20mm,20mm}]{geometry}
%\usepackage{multirow}

\begin{document}
\section{Program}
\subsection{Datum}
42.42.2042

\subsection{Zeitablauf}
\begin{tabular}{ll}
1815-1820 & Moderator grüsst und erklärt das Thema\\
1820-1920 & Diskussion der Teilnehmer\\
1920-1950 & Fragen/Anregungen aus dem Publikum\\
1950-2000 & Resumee der Teilnehmer\\
2000-2010 & Resumee des Moderators und Verabschiedung\\
2010-         & Speiss und Trank für die Angeregte Nachbearbeitung\\
\end{tabular}

\subsection{Adressaten}
\begin{itemize}
 \item Studierende der ZHAW/HSZT
 \item Leser der NZZ und des Tages Anzeiger
\end{itemize}


\subsection{Namen der Referenten}
\begin{itemize}
 \item Wird durch Linda und Oliver ausgefüllt
\end{itemize}

\subsection{Ort}
Aula, Uni Zürich (weitere Säle können nach Bedarf durch Videokonferenz zugeschalten werden)


\end{document}